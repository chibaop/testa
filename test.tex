\documentclass[11pt, oneside]{article}   	% use "amsart" instead of "article" for AMSLaTeX format
\usepackage{geometry}                		% See geometry.pdf to learn the layout options. There are lots.
\geometry{letterpaper}                   		% ... or a4paper or a5paper or ... 
%\geometry{landscape}                		% Activate for rotated page geometry
%\usepackage[parfill]{parskip}    		% Activate to begin paragraphs with an empty line rather than an indent
\usepackage{graphicx}				% Use pdf, png, jpg, or eps§ with pdflatex; use eps in DVI mode
\usepackage{multicol}								% TeX will automatically convert eps --> pdf in pdflatex		
\usepackage{amssymb}
\usepackage{fancybox}
\usepackage{ascmac}
%SetFonts

%SetFonts


\title{LaTeXの使い方}
\author{千葉 大輝}
%\date{}							% Activate to display a given date or no date

\begin{document}
\maketitle
kokono
\newpage
\section{はじめに}
LaTexの使い方を学ぶと同時に覚えておく為の、自分に分かりやすいように示しておく。基本的なコマンド等に関しては、説明・解説を行わない。卒業論文等に必須のLaTexのコマンド習得のためである。
\subsection{続いて}
コマンド自体の記入は行わない。他にもコマンド自体の練習としての利用にこの書類を使用する。
\section{段組を2段にする方法}
最初は、コマンド自体を使用しない場合においての、書文を見ると段組が2段になっていないことがわかる。
\begin{multicols}{2}
このように2つの段組を利用する場合には、multicolsを使用する。必ずusepackageの指定等を行うように。
左側の文章と右側の文章の文字間隔等を利用する場合にはtwocolumnを使用することをおすすめする。
\end{multicols}

\section{表}
表の組み方には、複数の行選択等のやり方が存在する。
\subsection{ラインなし}
最も一般的であるライン無しの表の組み方のコマンドを示す。
\begin{table}[h]
\begin{center}
\begin{tabular}{lcr}
cell1&cell2\\
cell3&cell4\\
\end{tabular}
\caption{ラインなしの表}
\end{center}
\end{table}

\subsection{ラインあり}
次に一般的であるラインなしのテーブルの組み方を示す。
\begin{table}[h]
\begin{center}
\begin{tabular}{|l|c|r|}
\hline
cell1 & cell2 & cell3 \\ \hline\hline
cell4 & cell5 & cell6 \\ \hline
cell7 & cell8 & cell9 \\ 
\hline
\end{tabular}
\caption{ラインありの表}
\end{center}
\end{table}

\newpage
\subsection{表を並べる方法}
\begin{table}[htbp]
\begin{center}
\begin{tabular}{cc}
\begin{minipage}{0.3\hsize}
\begin{center}
\begin{tabular}{|c|c|c|}
\hline
a & a & a \\ \hline
a & a & a \\ \hline
a & a & a \\
\hline
\end{tabular}
\caption{aが9つ}
\end{center}
\end{minipage}

\begin{minipage}{0.3\hsize}
\begin{center}
\begin{tabular}{|c|c|c|}
\hline
b & b & b \\ \hline
b & b & b \\ \hline
b & b & b \\
\hline
\end{tabular}
\caption{bが9つ}
\end{center}
\end{minipage}
\end{tabular}
\end{center}
\end{table}

\section{箇条書き}
箇条書きについてのコマンドを示す。
\subsection{記号付き箇条書き}
\begin{itemize}
\item 記号付き箇条書き
\item 番号付き箇条書き
\item 見出しつき箇条書き
\end{itemize}

\subsection{番号付き箇条書き}
\begin{enumerate}
\item  記号付き箇条書き
\item 番号付き箇条書き
\item 見出し付き箇条書き
\end{enumerate}

\subsection{見出し付き箇条書き}
\begin{description}
\item[その1] 記号付き箇条書き
\item[その2] 番号付き箇条書き
\item[その3] 見出し付き箇条書き
\end{description}

\section{枠で囲む}
コメント等を枠で囲む際の使用方法をここに記述する。
\subsection{タイトルなし}
fancyboxというマクロを使用するため、usepackageへの書き込みを忘れないようにする。
\fbox{普通の枠}
\doublebox{2重の枠}
\ovalbox{丸みのある枠}
\Ovalbox{丸みのある強調枠}
\shadowbox{影付きの枠}

\subsection{タイトル付きでの枠}
ascmacというマクロを使用するため、usepackageへの書き込みを忘れないようにする。
\begin{itembox}{種類}
\begin{center}
普通の枠\\
2重の枠\\
丸みのある枠\\
丸みのある強調枠\\
影付きの枠
\end{center}
\end{itembox}

\section{数式}
数式をlatexで出力するには、コマンドを必要とする。
\subsection{上付き文字、下付き文字}
\begin{eqnarray}
    G_{-1}(x,y)=x^{n-1}+x^n-1
\end{eqnarray}

\end{document}  






